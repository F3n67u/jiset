\section{Related Work}\label{sec:related}
Our technique is closely related to three fields: parser generation,
automatic semantics extraction, and formal semantics of JavaScript.

\textbf{Parser Generation:} From Packrat parsing~\cite{packrat} with
PEG~\cite{peg}, recursive-descent
parsers with backtracking support linear-time parsing.  However, it
has the fundamental problem of ordered choices: \( ab \) is silently
unmatched with \( a ~/~ ab \).  While Generalized LL (GLL) parsing~\cite{gll}
is basically recursive-descent with backtracking that can support general
context-free grammars even in the presence of ambiguous grammars,
its worst-time complexity is \( O(n^3) \) for the input size \( n \) and it
does not support context-sensitive features.  Unlike GLL parsing, our lookahead
parsing is applicable for JavaScript parsers with context sensitive features
such as positive/negative lookaheads. Moreover, the complexity of lookahead
parsing is \( O(k \cdot n) \) for the constant
number of tokens \( k \). We experimentally showed that it can generate parsers
for the most recent four versions of ECMAScript.

\textbf{Automatic Semantics Extraction:} The closest work to ours is the formal
semantics extraction for x86~\cite{extract-x86} and ARM~\cite{extract-arm}.
They utilized complex Natural Language Processing (NLP) and Machine Learning
(ML) to extract formal semantics of small-sized low-level assembly languages.
Another related work is \citet{javadoc}'s automatic model generation, which
generates Java code from Javadoc comments for API functions.  Using NLP techniques
and heuristic methods, it produces candidate code and removes wrong ones
by testing them with actual implementation. Unlike their approach, we
introduce a semi-automatic synthesis using general compile rules that
represent common writing patterns of specifications.  The extracted
semantics by \(\tool\) is also executable, which allows to bridge gaps
between the specification written in a natural language and executable
tests.

Several approaches defined the formal semantics of JavaScript.
\citet{lambdajs} defined a core calculus of JavaScript expressing non-core
features using desugaring, but its correspondence with ECMAScript is
not obvious.
KJS~\cite{kjs} and JaVerT~\cite{javert} defined the JavaScript semantics
by manually converting ECMAScript to their own formal languages.  KJS mapped
ES5.1 in the K framework~\cite{kframework} and JaVerT converts the
specification to their own IR.  However, they all target only ES5.1 or former and
they do not provide any solution for annual updates of ECMAScript. Our approach
provides a mechanized framework to synthesize JavaScript parsers and to
automatically extract semantics using a rule-based compilation technique, which
significantly reduces human efforts.
