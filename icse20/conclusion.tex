\section{Conclusion}\label{sec:conclude}
The annual updates of the ECMAScript specification makes it difficult to
build program analysis or formal verification of JavaScript due to the
required human efforts in modeling a moving target.  In this paper,
we proposed a tool \( \tool \), which \textit{automatically} extracts the
syntax and semantics specified in the ECMAScript specification.  The tool
generates a parser from a language grammar written in \( \bnfes \), and
compiles abstract algorithms written in English to \( \ires \) functions.
The only parts that require manual efforts are compile rules and
language-specific global setting, which we provide for ECMAScript
as an example case study.  We also support \textsf{Fault Localizer} to
adapt updates of the ECMAScript specification by suggesting new rules.
We evaluated \( \tool \) by applying it to ECMAScript 2020 with Test262.
The extracted semantics passed 19,220 of 19,282 applicable tests in Test262.
The tool also detected three confirmed errors from the specification.
Moreover, we showed that the tool is also applicable to an incomplete
language proposal. The implementation of the tool is publicly available
as an open-source project.
