\section{Related Work}\label{sec:related}

Our technique is closely related by these fields: parser generation, automatic model generation, and formal semantics of JavaScript.

From Packrat parsing~\cite{packrat} with parsing expression grammars (PEG (PEG))~\cite{peg},
the recursive-descent parsers with backtracking supports linear-time parsing.
However, it suffers from the fundamental problem of ordered choices;
\( ab \) is silently unmatched in \( a \mid ab \) quirk.
The Generalized LL (GLL) parsing~\cite{gll} is basically recursive-descent with backtracking
but it could support general context-free grammars (CFG) even including ambiguous grammars.
However, it takes \( O(n^3) \) parsing time when the input size is \( n \) in worst cases.
While it could be reduced into almost linear time on most real grammars,
it should keep all intermediate possible parsing results during parsing mechanisms.
Unlike GLL parsing, our lookahead parsing is not applicable into general context-free grammar.
However, we experimentally show that JavaScript parsers could be generated via lookahead parsers.
Lookahead parsers take \( O(kn) \) where \( n \) is the length of inputs and \( k \)
is the number of tokens. In JavaScript parser, it has only constant number of tokens
thus the generated JavaScript parsers with lookahead parsers take linear time.

There is similar problems in many fields. Not only language semantics, many specification is not constructed formally, so many researches are trying to
extract formal semantics of target specification. To reduce manual efforts to get models of target system, there are few studies about automatic generation of
formal semantics from specification. \cite{javadoc} is closely related work of ours, which uses Javadoc comments of target API function with NLP techniques and various
heuristic methods, to generate java code which correctly models target. They produces several candidate codes, and test their codes with actual implementation of API functions
to find correct modeling. Unlike their approaches, we provides specification translation assistant to support future language update, and to provide correct meaning of specification, which also can applicable when
testing for each primitive is unavailable so we cannot decide correct semantics automatically.

% TODO pseudo code의 형태와 natural language의 결합이다.
% TODO moudular test로 guide를 받을 수 없어서 user defined rule을 사용. 대신, statistical analysis를 기반으로 guide해줌

% \begin{itemize}
%   \item Automatic Model Generation from Documentation for Java API Functions (ICSE'16)~\cite{javadoc}
% \end{itemize}



% Building formal semantics of language is important for program verification and static analysis. JavaScript is most widely-used language in
% web browser, but its complex behavior makes confusion to developer and program analysis designer. So there are many researches to define formal semantics of JavaScript and
% using it to formal verification or program analysis.
There are also many researches to define formal semantics of JavaScript. \cite{lambdajs} define the core calculus of JavaScript, and express rest of the features as desugaring. Their method can
work for subset of JavaScript, but their formalization is not directly mapped to specification document. So it requires deep understanding of current and upcoming specification,
which needs lots of effort. \cite{kjs} and \cite{javert} defines formal semantics of JavaScript based on ECMAScript specification, and they manually convert from specification
to formal languages. KJS mapped from ECMAScript 5 specification to K framework, which can be used in program verification and analysis. Similarly, JaVerT coverts specification
to their own language (called JSIL) to make further analysis. Compared to these approaches, we suggests to generate converter from specification to intermediate language, with
conversion assistant and layered translation technique to minimize human effort and to apply for further update or even unconfirmed proposal of specification.

% \begin{itemize}
%   \item JaVerT: JavaScript verification toolchain (POPL'18)~\cite{javert}
%   \item KJS: A complete formal semantics of JavaScript (PLDI'15)~\cite{kjs}
%   \item The Essence of JavaScript (ECOOP'10)~\cite{lambdajs}
%   \item A Trusted Mechanised JavaScript Specification (POPL'14)~\cite{jscert}
% \end{itemize}
