\begin{abstract}
  JavaScript is one of mainstream programming languages for client-side
  programming, server-side programming, and even embedded systems.  Thus,
  various JavaScript engines are developed and maintained in diverse fields, and
  they must conform to the syntax and semantics described in ECMAScript, the
  standard specification of JavaScript.  Since an incorrect description in
  ECMAScript can lead to the wrong JavaScript engine implementations, checking
  the correctness of ECMAScript is essential.  However, all the specification
  updates are currently manually reviewed by the Ecma Technical Committee 39
  (TC39) without any automated tools.  Moreover, in late 2014, the committee
  announced yearly release cadence and open development process of ECMAScript to
  adapt to eolving development environments quickly.  Because of such frequent
  updates, checking the correctness of ECMAScript becomes more labor-intensive
  and error-prone.

  To alleviate the problem, we propose $\tool$, a JavaScript Specification Type
  Analyzer using Refinement.  It is the first tool that performs \textit{type
  analysis} on JavaScript specifications and detects specification bugs using a
  \textit{bug detector}.  For a given specification, $\tool$ first compiles each
  abstract algorithm written in a structured natural language to a corresponding
  function in $\ires$, an untyped intermediate representation for ECMAScript.
  Then, it performs type analysis for compiled functions with specification
  types defined in ECMAScript.  Based on the result of type analysis, $\tool$
  detects specification bugs using a bug detector consisting of four checkers.
  Moreover, to increase precision of type analysis, we present
  \textit{condition-based refinement} for type analysis, which prunes out
  infeasible abstract states using conditions of assertions and branches.  We
  evaluated $\tool$ with all \inred{XXX} versions in the official ECMAScript
  repository for the recent \inred{three years from 2018 to 2021}.  Our tool
  detected \inred{XXX} specification bugs including only \inred{XXX} false
  alarms caused by the imprecision of type analysis in \inred{XX} minutes on
  average.  Among \inred{XXX} true alarms, \inred{XXX} bugs are resolved after
  existing for \inred{XXX} days on average and \inred{XX} bugs still exist in
  the latest version of ECMAScript for \inred{XXX} days.  We reported all
  \inred{XX} newly found bugs and they are confirmed by the committee.
\end{abstract}
