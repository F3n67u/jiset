\section{Related Work}\label{sec:related}
Our technique is closely related to three fields: parser generation,
automatic model generation, and formal semantics of JavaScript.

From Packrat parsing~\cite{packrat} with PEG~\cite{peg}, recursive-descent
parsers with backtracking support linear-time parsing.  However, it
has the fundamental problem of ordered choices: \( ab \) is silently
unmatched with \( a \mid ab \).  While Generalized LL (GLL) parsing~\cite{gll}
is basically recursive-descent with backtracking that can support general
context-free grammars (CFGs) even in the presence of ambiguous grammars,
its worst-time complexity is \( O(n^3) \) for the input size \( n \).
Unlike GLL parsing, our lookahead parsing is not applicable to general CFGs,
but we experimentally showed that it can generate JavaScript parsers.
The complexity of lookahead parsing is \( O(k \cdot n) \) for the input size
\( n \) and the constant number of tokens \( k \).

In the field of automatic model generation, the closest related work to
ours is \citet{javadoc}'s, which generates Java code from Javadoc comments
for API functions.  Using NLP techniques and various heuristic methods,
it produces several candidate code tests them with actual implementation
of the API functions to find correct modeling.  Unlike their approach, we
introduce a fault localization algorithm to support future language
updates and to provide correct meaning of specification. Our tool is also
applicable for testing even when the actual implementation is not available.

% TODO pseudo code의 형태와 natural language의 결합이다.
% TODO moudular test로 guide를 받을 수 없어서 user defined rule을 사용. 대신, statistical analysis를 기반으로 guide해줌

% \begin{itemize}
%   \item Automatic Model Generation from Documentation for Java API Functions (ICSE'16)~\cite{javadoc}
% \end{itemize}



% Building formal semantics of language is important for program verification and static analysis. JavaScript is most widely-used language in
% web browser, but its complex behavior makes confusion to developer and program analysis designer. So there are many researches to define formal semantics of JavaScript and
% using it to formal verification or program analysis.

Several approaches have defined the formal semantics of JavaScript.
\citet{lambdajs} defined a core calculus of JavaScript, and expressed the
rest as desugaring.  While it presents the core JavaScript semantics in
an extension of the $\lambda$ calculus, it does not show the semantics
correspondence defined in the ECMAScript specification.  KJS~\cite{kjs} and
JaVerT~\cite{javert} defined formal semantics of JavaScript by manually converting
the ECMAScript specification to their own formal languages.  KJS mapped
ECMAScript 5.1 in the K framework~\cite{kframework}, which can be used
for program verification and analysis.  Similarly, JaVerT converts the
specification to their own IR called JSIL to make further analysis.
Our approach provides a mechanized framework to synthesize syntactic parsers and
to automatically extract semantics
using a rule-based compilation technique, which significantly reduces human
efforts.

% \begin{itemize}
%   \item JaVerT: JavaScript verification toolchain (POPL'18)~\cite{javert}
%   \item KJS: A complete formal semantics of JavaScript (PLDI'15)~\cite{kjs}
%   \item The Essence of JavaScript (ECOOP'10)~\cite{lambdajs}
%   \item A Trusted Mechanised JavaScript Specification (POPL'14)~\cite{jscert}
% \end{itemize}
