\begin{abstract}
  JavaScript is one of mainstream programming languages not only in client-side
  programming but also in server-side programming and even embedded systems.
  Thus, various JavaScript engines are developed and maintained in various
  fields and they must conform to the syntax and semantics described in
  ECMAScript, the standard specification of JavaScript.  Since an incorrect
  description in ECMAScript can lead to the wrong implementation of JavaScript
  engines, it is important to check the correctness of ECMAScript.  However, all
  the specification updates are currently manually checked by the Ecma Technical
  Committee 39 (TC39) without any automated tools.  Moreover, in late 2014, the
  committee announced yearly release cadence and open development process of
  ECMAScript to quickly adapt to an evolving development environment.  Thus, the
  manual checking becomes more labor-intensive and error-prone.

  To alleviate this problem, we propose $\tool$, a JavaScript Specification Type
  Analyzer using Refinement.  It is the first tool that performs \textit{type
  analysis} on JavaScript specifications and detects specification bugs using
  \textit{bug checkers}.  For a given specification, our tool first compiles
  each abstract algorithm written in a structured natural language to the
  corresponding function of $\ires$, which is an untyped intermediate
  representation for ECMAScript.  Then, it performs type analysis for compiled
  functions with specification types defined in the given ECMAScript.  Based on
  the result of type analysis, $\tool$ detects specification bugs using four
  different bug checkers.  Moreover, to increase precision of type analysis, we
  also present \textit{refinement} for type analysis, which prunes out
  infeasible abstract states by using conditions in assertions and branches.
  We evaluated $\tool$ with all 787 different versions of ECMAScript existed in
  the official repository for the recent three years (from 20 March 2018 to 9
  March 2021).  Our tool detected \inred{XXX} specification bugs including only
  \inred{XXX} false alarms caused by the imprecision of type analysis.  Among
  \inred{XXX} true alarms, \inred{XXX} bugs are resolved after existing for
  \inred{XXX} days on average and \inred{XX} bugs still exist in the latest
  version of ECMAScript for \inred{XXX} days.  We reported all \inred{XX} newly
  found bugs and they are confirmed by the committee.
\end{abstract}
