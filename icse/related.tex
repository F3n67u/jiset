\section{Related Work}

Our technique is closely related by these fields: Parser generation, Automatic model generation, and Formal semantics of JavaScript.
\subsection{Parser Generation}
\begin{itemize}
  \item Packrat parsing:: simple, powerful, lazy, linear time, functional pearl (ICFP'02)~\cite{packrat}
    % left recursion 잘 안됨 / memoization으로 memory가 엄청 커짐
    % GLL - follow가 아닌 prediction으로 사용함 / 이 방법을 적용하면 중간중간 발생하는 nondeterministic한 경우를 모두 keeping해야됨.
    % context-sensitive token이기에 미리 parsing을 한 token이 존재하지 않기 때문에 그 다음 token이 뭔지 알 수 없어서 prediction도 하지 못함
\end{itemize}

\subsection{Automatic Model Generation}
There is similiar problems in many fields. Not only langauge semantics, many specification is not constructed formally, so many researcheres are trying to
extract formal semantics of target specification. To reduce manual efforts to get models of target system, there are few studies about automatic generation of
formal semantics from specification. \cite{javadoc} is closely related work of ours, which uses javadoc comments of target API function with NLP techniques and various
heuristic methods, to generate java code which correctly models target. They produces several candidate codes, and test their codes with actual implementation of API functions
to find correct modeling. Unlike their approaches, we provides specification translation assistant to support future language update, and to provide correct meaning of specification, which also can applicable when
testing for each primitive is unavailable so we cannot decide correct semantics automatically.

% \begin{itemize}
%  \item Automatic Model Generation from Documentation for Java API Functions (ICSE'16)~\cite{javadoc}
% \end{itemize}

\subsection{Formal Semantics of JavaScript}
Building formal semantics of JavaScript language is 
\begin{itemize}
  \item JsVert: \cite{javert}
  \item KJS: A complete formal semantics of JavaScript (PLDI'15)~\cite{kjs}
  \item The Essence of JavaScript (ECOOP'10)~\cite{lambdajs}
  \item A Trusted Mechanised JavaScript Specification (POPL'14)~\cite{jscert}
\end{itemize}
