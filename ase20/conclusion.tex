\section{Conclusion}\label{sec:conclude}

The annual updates of the ECMAScript makes it difficult to build program
analysis or formal verification of JavaScript due to the required human efforts
in modeling a moving target.  In this paper, we proposed a tool \( \tool \),
which \textit{automatically} extracts the syntax and semantics specified in
ECMAScript.  The tool generates a parser from a language grammar written in \(
\bnfes \), and an AST-IR translator from abstract algorithms written in
English.  The tool automatically extract all syntax and \inred{XX.XX\%} of
semantics for the most recent four versions of ECMAScript (ES7 to ES10).  We
evaluated the correctness of the tool by testing the extracted semantics from
ES10 with Test262, the official conformance suite.  Based on \inred{X,XXX}
failed tests, we found eight specification errors in ES10. After resolving
them, the extracted semantics passed all \inred{XX,XXX} applicable tests.
Moreover, we showed that \( \tool \) is also forward compatible for the future
language features by applying it to nine proposals, and we found four
specification errors in the proposal for BigInt.  We believe that it could
dramatically reduce the human efforts to define formal semantics of ECMAScript
and also bridge gaps between ECMAScript and Test262.
