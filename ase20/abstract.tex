\begin{abstract}
JavaScript was initially designed for client-side programming in web browsers,
but its engine is now embedded in various kinds of host software.  Despite the
popularity, since the JavaScript semantics is complex especially due to its
dynamic nature, understanding and reasoning about JavaScript programs are
challenging tasks.  Thus, researchers have proposed several attempts to define
the formal semantics of JavaScript based on ECMAScript, the official JavaScript
specification.  However, the existing approaches are manual, labor-intensive,
and error-prone and all of their formal semantics target ECMAScript 5.1 (ES5.1, 2011)
or its former versions.  Therefore, they are not suitable for understanding
\textit{modern JavaScript} language features introduced since
ECMAScript 6 (ES6, 2015).  Moreover, ECMAScript has been annually updated since ES6,
which already made five releases after ES5.1.

To alleviate the problem, we propose \( \tool \), a JavaScript IR-based
Semantics Extraction Toolchain.  It is the first tool that \textit{automatically
synthesizes} parsers and AST-IR translators directly from a given
language specification, ECMAScript.
For syntax, we develop a parser generation technique with \textit{lookahead
parsing} for \( \bnfes \), a variant of the extended BNF used in ECMAScript.
For semantics, \( \tool \) synthesizes AST-IR translators using \textit{forward
compatible} rule-based compilation.  \textit{Compile rules} describe how to
convert each step of abstract algorithms written in a structured natural language
into \( \ires \), an Intermediate Representation that we designed for
ECMAScript.  For the four most recent ECMAScript versions, \( \tool \) automatically
synthesized parsers for all versions, and compiled 95.03\% of the
algorithm steps on average.  After we complete the missing parts manually,
the extracted core semantics of the latest ECMAScript (ES10, 2019) passed all
18,064 applicable tests.  Using this \textit{first formal semantics of modern
JavaScript}, we found nine specification errors in ES10, which were all
confirmed by the Ecma Technical Committee 39.  Furthermore, we showed that \(
\tool \) is \textit{forward compatible} by applying it to nine feature
proposals ready for inclusion in the next ECMAScript, which let us find four
errors in the BigInt proposal.
\end{abstract}
